\documentclass[11pt]{article}

\setlength{\topmargin}{0pt}
\setlength{\textheight}{9in}
\setlength{\headheight}{0pt}
\setlength{\headsep}{0pt}
\setlength{\oddsidemargin}{0.25in}
\setlength{\textwidth}{6in}
\pagestyle{plain}

\begin{document}

\thispagestyle{empty}

%%%%%%%%%%%%%%%%%%%%%%%%%%%%%%%%%%%%%%%%%%%%%%%%%%%%%%%%%%%%%%%%%%
\raisebox{0.6in}[0in]{\makebox[\textwidth][r]{\it
 Course Introduction}}
\vspace{-0.7in}
%%%%%%%%%%%%%%%%%%%%%%%%%%%%%%%%%%%%%%%%%%%%%%%%%%%%%%%%%%%%%%%%%%

\begin{center}
\bf\large IST718 Big Data Analysis
\end{center}

\noindent
Lecturer:                Prof. Daniel Acuna
\hfill
Lecture \#               1
\\
Scribe:                  Lizhen Liang \& Yimin Xiao
\hfill
                         1/15/2019

\noindent
\rule{\textwidth}{1pt}

\medskip

%%%%%%%%%%%%%%%%%%%%%%%%%%%%%%%%%%%%%%%%%%%%%%%%%%%%%%%%%%%%%%%%
%% body of scribe notes goes here
%%%%%%%%%%%%%%%%%%%%%%%%%%%%%%%%%%%%%%%%%%%%%%%%%%%%%%%%%%%%%%%%
\section{What is data science?}
According to the Venn Diagram, data science is the overlapping area of hacking skills, math \& statistics knowledge and domain knowledge

\subsection{The "classic" and "new" kinds of data science}

The “classic” kind of data science means a bunch of experts in charge of creating relatively small and interpretable models and providing features to describe the dataset or do predictions. The accuracy of the models created by “classic” data scientist cannot be proved.

By contrast, the “new” kind of data science need no expert. Models are generalized and takes data with features of low level. For example, in old days, experts generate features that can describe transaction data, while nowadays experts simply use the raw transaction data as the input data for the model.

Models that are used for the “new” kind of data science are generally black boxes: they are hard to explain.

\section{What is big data?}
According to a study given by Andrew Ng and Michael I. Jordon, when the dataset is small, a simpler model has smaller error rate than a more complicated model. But when the size of a dataset gets bigger, the error rate of the more complicated model is smaller than the simpler model.

Big data means data with large volume, fast velocity and complex data variety(“three V’s ” for big data).

Big data is usually stored in a distributed system with no structure.

\section{Example for big data applications}

\begin{description}
\item[Recommendation System]
\item[Activity Recognition]
\end{description}



%%%%%%%%%%%%%%%%%%%%%%%%%%%%%%%%%%%%%%%%%%%%%%%%%%%%%%%%%%%%%%%%

\end{document}
